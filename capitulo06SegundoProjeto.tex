\chapter{Segundo Projeto: Sistema para Locação de DVDs v1.0}\label{cap:segundoProjeto}
\epigraph{``\textit{Só se conhece o que se pratica}''.}{Barão de Montesquieu}

\lettrine[lines=4, lhang=0.1, lraise=0, loversize=0.2, findent=0.1em]{\textcolor{corTema}{N}}{ESTE} Capítulo aplicaremos o conhecimento adquirido até o momento na construção de uma aplicação Web em Java.


\section{Introdução}

Neste Capítulo será apresentada uma série de requisitos que devem ser usados para criar uma aplicação Web da mesma forma que fizemos no Capítulo~\ref{cap:primeiroProjeto}. Tudo que será requisitado estará baseado no que já aprendemos, sendo assim, todas as funcionalidades requeridas poderão ser implementadas com recursos já vistos no Capítulo~\ref{cap:primeiroProjeto}. Note que apesar do projeto ser intitulado como um sistema para locação de DVDs, por enquanto a implementação da locação em si será deixada de lado, pois a faremos no projeto do Capítulo~\ref{cap:quartoProjeto}. Isso se dá porque ainda precisamos aprender mais algumas coisas, principalmente do lado do cliente, para sermos capazes de implementar cadastros que lidem com relacionados muitos para muitos.


\section{Apresentação dos Requisitos}

Você foi contratado para criar um sistema para controle de cadastro de DVDs. Esse sistema irá manter apenas o cadastro de DVDs e não irá gerenciar a locação dos mesmos. Os dados que deverão ser mantidos para todos os DVDs são: título, ano de lançamento, ator principal, ator coadjuvante, data de lançamento, duração em minutos, gênero e classificação etária. Os dados dos atores, dos gêneros e das classificações etárias deverão ser mantidos através de cadastros específicos. Um ator deve ter um nome, um sobrenome e uma data de estreia (indica quando foi o primeiro filme do ator). Um gênero deve ter uma descrição. Uma classificação etária deve ter também apenas uma descrição. Cada um dos cadastros (DVD, ator, gênero e classificação etária), deve conter as funcionalidades de criar, alterar e excluir um determinado registro. A página principal da aplicação deve conter um link para cada tipo de cadastro.


\section{Desenvolvimento do Projeto}

O projeto Web que deverá ser criado deve ter o nome de ``LocacaoDVDs''. Configure o projeto para conter as bibliotecas internamente. O pacote de código-fonte base do projeto deve ter o nome de ``\texttt{locacaodvds}''. A estrutura do projeto deve ser igual à estrutura do projeto criado no Capítulo~\ref{cap:primeiroProjeto}, sendo que, obviamente, o nome das classes e suas respectivas implementações serão diferentes do projeto daquele Capítulo. A base de dados com as tabelas das entidades obtidas a partir da análise dos requisitos na seção anterior deve ter o nome de ``\texttt{locacao\_dvds}''. Não se esqueça de que cada entidade deverá conter um identificador. As páginas da aplicação deverão ter sua aparência configurada usando uma folha de estilos, da mesma forma que fizemos no projeto do Capítulo~\ref{cap:primeiroProjeto}. Tente personalizar os estilos, mudando as cores etc. Você está livre para usar imagens ou qualquer outro artifício que julgar interessante para mudar a aparência da sua aplicação.


\section{Resumo}

Neste Capítulo foi requisitado que você implementasse uma aplicação Web em Java para gerenciar o cadastro de DVDs de uma locadora (ainda sem a locação), por isso, não há atividades a serem realizadas.
