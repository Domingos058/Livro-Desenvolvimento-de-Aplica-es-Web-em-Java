\chapter{\textit{Frameworks} MVC e de Persistência}\label{cap:frameworksPersistencia}
\epigraph{``\textit{A persistência é o caminho do êxito}''.}{Charles Chaplin}
\epigraph{``\textit{Não extingua sua inspiração e sua imaginação; não se torne o escravo do seu modelo}''.}{Vincent van Gogh}

\lettrine[lines=4, lhang=0.1, lraise=0, loversize=0.2, findent=0.1em]{\textcolor{corTema}{N}}{ESTE} Capítulo reconstruiremos o ``Sistema de Venda de Produtos'' do Capítulo~\ref{cap:terceiroProjeto} utilizando diversos \textit{frameworks} que tornarão nosso trabalho menos tedioso e mais direto ao ponto!

\vfill

\section{Introdução}

Antes de começarmos a trabalhar, precisamos configurar nosso ambiente de desenvolvimento. Como lidaremos com toda a \textit{stack} de \textit{frameworks} e bibliotecas do Spring, usaremos a ferramenta oficial deles para nos ajudar, a Spring Tool Suite. Eu particularmente não sou muito fã, pois ela é baseada na IDE Eclipse que, na minha opinião, é extremamente burocrática e instável, mas como ela é adotada como o padrão da indústria, uma hora ou outra acabamos ter que a adotar e, como queremos ver qual é que é a do mundo real, vamos utilizá-la.

INSTALAçÂO da FERRAMENTA

falar dos workspaces

mostrar interface


\section{Spring Boot}

Já sabemos o que é o Spring Boot e agora o usaremos para nos ajudar a configurar e a usar os seguintes frameworks:

\begin{itemize}
    \item \textbf{Spring Boot DevTools:} auxília no desenvolvimento de aplicações usando Spring Boot;
    \item \textbf{Spring Framework:} \textit{framework} que possibilita a injeção de dependências e a inversão de controle;
    \item \textbf{Spring Data JPA:} \textit{framework} para persistência de dados usando Hibernate e a Java Persistence API (JPA);
    \item \textbf{Spring Validation:} permite a validação de objetos que serão gerenciados pela aplicação (já fizemos algo parecido);
    \item \textbf{Spring Web:} construção de aplicações Web usando o padrão de projeto MVC através do \textit{framework} Spring MVC, além da criação de Web Services RESTful;
    \item \textbf{Lombok:} é uma biblioteca de anotações que nos ajuda na criação automática de código padronizado como getters, setters, construtores etc;
    \item \textbf{Thymeleaf:} é uma \textit{template engine} que permite a definição de modelos para a criação de interfaces gráficas usando HTML, diminuindo a duplicidade de código e facilitando a manutenção do código.
\end{itemize}

CRIANDO PROJETO

explicar estrutura do projeto

como executar

como associar ao navegador com LiveReload

arquivo de configurações do Spring Boot


\section{Hibernate, JPA, Validações e Lombok}

criação das entidades
carga inicial no banco


\section{Spring MVC}

criação dos controladores e páginas com thymeleaf


\subsection{Outros \textit{Frameworks} MVC}

%Falar brevemente de: JavaServer Faces (JSF), Struts, VRaptor etc.


\section{\textit{Web Services REST}}

criação de controladores rest e consumir API usando Javascript.


\section{Resumo}

\section{Exercícios}

\section{Projetos}
