\chapter{JavaScript}\label{cap:javaScript}
\epigraph{``\textit{A vida vai ficando cada vez mais dura perto do topo}''.}{Friedrich Nietzsche}

\lettrine[lines=4, lhang=0.1, lraise=0, loversize=0.2, findent=0.1em]{\textcolor{corAzulTema}{N}}{ESTE} Capítulo temos com objetivo aprender as construções básicas da linguagem de programação JavaScript, vastamente utilizada no desenvolvimento de aplicações Web.

\section{Introdução}

Chegamos a talvez à cereja do bolo, ou à cereja do livro ou então à cereja do desenvolvimento para Web: a linguagem de \textit{script} JavaScript. A primeira coisa que precisamos deixar claro é que Java e JavaScript são duas linguagens diferentes, com sintaxe baseada em C e com construções similares, mas o comum entre as duas para por aqui. A linguagem JavaScript não é uma linguagem orientada a objetos, mas sim baseada em protótipos. Nesse tipo de linguagem não existem classes, mas apenas objetos. Novos objetos são criados a partir de cópias de objetos existentes. O JavaScript moderno, baseado no padrão ECMAScript, possui algumas construções que lembram a orientação a objetos como classes e herança de entre classes, mas tudo isso é \textit{syntax sugar} para simplificar coisas que já existiam anteriormente. Outras construções também são suportadas na linguagem como as funções de primeira classe etc.

Neste Capítulo não entraremos em detalhes sobre a história da linguagem e da sua evolução, mas sim no que eu acho útil e fundamental para que possamos começar a usar a linguagem. Iremos ver uma visão geral da linguagem como declaração de variáveis, manipulação do \textit{Document Object Model} (DOM) e requisições assíncronas. Durante o Capítulo serão apresentadas inúmeras caixas do tipo ``Saiba Mais'' com \textit{links} úteis. A maioria desses links serão da \textit{Mozilla Developer Network} (\url{https://developer.mozilla.org/}), uma referência confiável e oficial da maioria, senão todas, tecnologias para Web. Sempre fornecerei os \textit{links} da versão em inglês do site, mas se quiser, você pode verificar a versão em português clicando no botão ``\textit{Change language}'' presente em todas as páginas do site. Sinceramente recomendo a leitura em inglês, pois o texto sempre estará completo, atualizado com a terminologia correta.

\begin{saibaMais}
    Quer conhecer um pouco mais da história e dos detalhes da linguagem JavaScript? Veja o \textit{link} \url{https://developer.mozilla.org/en-US/docs/Web/JavaScript}.
\end{saibaMais}

\begin{saibaMais}
    A referência da linguagem JavaScript pode ser acessada pelo \textit{link} \url{https://developer.mozilla.org/en-US/docs/Web/JavaScript/Reference}.
\end{saibaMais}

\begin{saibaMais}
    As documentações e referências das tecnologias e APIs usadas para o desenvolvimento para Web podem ser acessadas pelo \textit{link} \url{https://developer.mozilla.org/en-US/docs/Web/API}.
\end{saibaMais}

Antes de começarmos a falar do JavaScript propriamente dito, vamos montar nosso palco, que é um projeto Java para Web.

Listagem~\thechapter.\ref{listagem:projetos/capitulo07/ExemplosEmJavaScript/web/index.jsp}

\htmlCode{Página principal da aplicação\newline%
Arquivo: \texttt{/index.jsp}}{projetos/capitulo07/ExemplosEmJavaScript/web/index.jsp}


Listagem~\thechapter.\ref{listagem:projetos/capitulo07/ExemplosEmJavaScript/src/java/exemplosemjavascript/pojo/Pessoa.java}.

\javaCode{Classe Pessoa\newline%
Arquivo: \texttt{exemplosemjavascript/pojo/Pessoa.java}}{projetos/capitulo07/ExemplosEmJavaScript/src/java/exemplosemjavascript/pojo/Pessoa.java}


Listagem~\thechapter.\ref{listagem:projetos/capitulo07/ExemplosEmJavaScript/src/java/exemplosemjavascript/servlets/CalculaTabuadaServlet.java}.

\javaCode{Servlet de tabuada\newline%
Arquivo: \texttt{exemplosemjavascript/servlets/CalculaTabuadaServlet.java}}{projetos/capitulo07/ExemplosEmJavaScript/src/java/exemplosemjavascript/servlets/CalculaTabuadaServlet.java}


Listagem~\thechapter.\ref{listagem:projetos/capitulo07/ExemplosEmJavaScript/src/java/exemplosemjavascript/servlets/ListagemPessoasServlet.java}.

\javaCode{Servlet de listagem de pessoas usando JSON\newline%
Arquivo: \texttt{exemplosemjavascript/servlets/ListagemPessoasServlet.java}}{projetos/capitulo07/ExemplosEmJavaScript/src/java/exemplosemjavascript/servlets/ListagemPessoasServlet.java}


\section{Funções de E/S e Operadores Aritméticos}

Listagem~\thechapter.\ref{listagem:projetos/capitulo07/ExemplosEmJavaScript/web/js/exemplo01.js}

\javaScriptCode{Exemplo 01\newline%
Arquivo: \texttt{/js/exemplo01.js}}{projetos/capitulo07/ExemplosEmJavaScript/web/js/exemplo01.js}



\section{Declarações de Variáveis e Suas Implicações}

Documentação: \url{https://developer.mozilla.org/en-US/docs/Web/JavaScript/Reference/Statements}

Listagem~\thechapter.\ref{listagem:projetos/capitulo07/ExemplosEmJavaScript/web/js/exemplo02.js}

\javaScriptCode{Exemplo 02\newline%
Arquivo: \texttt{/js/exemplo02.js}}{projetos/capitulo07/ExemplosEmJavaScript/web/js/exemplo02.js}



\section{Estruturas Condicionais, Operadores Relacionais e Lógicos}

lista completa de operadores
\url{https://developer.mozilla.org/en-US/docs/Web/JavaScript/Reference/Operators}

Listagem~\thechapter.\ref{listagem:projetos/capitulo07/ExemplosEmJavaScript/web/js/exemplo03.js}

\javaScriptCode{Exemplo 03\newline%
Arquivo: \texttt{/js/exemplo03.js}}{projetos/capitulo07/ExemplosEmJavaScript/web/js/exemplo03.js}



\section{Estruturas de Repetição, Arrays e Funções de Iteração}

documentação arrays:
\url{https://developer.mozilla.org/en-US/docs/Web/JavaScript/Reference/Global_Objects/Array}

Listagem~\thechapter.\ref{listagem:projetos/capitulo07/ExemplosEmJavaScript/web/js/exemplo04.js}

\javaScriptCode{Exemplo 04\newline%
Arquivo: \texttt{/js/exemplo04.js}}{projetos/capitulo07/ExemplosEmJavaScript/web/js/exemplo04.js}



\section{``Classes'', Objetos e JSON}

String: \url{https://developer.mozilla.org/en-US/docs/Web/JavaScript/Reference/Global_Objects/String }
JSON: \url{https://developer.mozilla.org/en-US/docs/Web/JavaScript/Reference/Global_Objects/JSON}

Listagem~\thechapter.\ref{listagem:projetos/capitulo07/ExemplosEmJavaScript/web/js/exemplo05.js}

\javaScriptCode{Exemplo 05\newline%
Arquivo: \texttt{/js/exemplo05.js}}{projetos/capitulo07/ExemplosEmJavaScript/web/js/exemplo05.js}



\section{Manipulação do DOM}

\url{https://developer.mozilla.org/en-US/docs/Web/API/Document_Object_Model}

\subsection{JavaScript Puro}

Listagem~\thechapter.\ref{listagem:projetos/capitulo07/ExemplosEmJavaScript/web/js/exemplo06.js}

\javaScriptCode{Exemplo 06\newline%
Arquivo: \texttt{/js/exemplo06.js}}{projetos/capitulo07/ExemplosEmJavaScript/web/js/exemplo06.js}



\subsection{Usando jQuery}

Listagem~\thechapter.\ref{listagem:projetos/capitulo07/ExemplosEmJavaScript/web/js/exemplo07.js}

\javaScriptCode{Exemplo 07\newline%
Arquivo: \texttt{/js/exemplo07.js}}{projetos/capitulo07/ExemplosEmJavaScript/web/js/exemplo07.js}



\section{Manipulação de Formulários}

Listagem~\thechapter.\ref{listagem:projetos/capitulo07/ExemplosEmJavaScript/web/js/exemplo08.js}

\javaScriptCode{Exemplo 08\newline%
Arquivo: \texttt{/js/exemplo08.js}}{projetos/capitulo07/ExemplosEmJavaScript/web/js/exemplo08.js}



\section{Eventos}

Listagem~\thechapter.\ref{listagem:projetos/capitulo07/ExemplosEmJavaScript/web/js/exemplo09.js}

\javaScriptCode{Exemplo 09\newline%
Arquivo: \texttt{/js/exemplo09.js}}{projetos/capitulo07/ExemplosEmJavaScript/web/js/exemplo09.js}



\section{Simulação Usando \textit{Canvas}}

Listagem~\thechapter.\ref{listagem:projetos/capitulo07/ExemplosEmJavaScript/web/js/exemplo10.js}

\javaScriptCode{Exemplo 10\newline%
Arquivo: \texttt{/js/exemplo10.js}}{projetos/capitulo07/ExemplosEmJavaScript/web/js/exemplo10.js}



\section{Requisições Assíncronas e Intercâmbio de Dados}


\subsection{AJAX com jQuery e com Fetch API}

Listagem~\thechapter.\ref{listagem:projetos/capitulo07/ExemplosEmJavaScript/web/js/exemplo11.js}

\javaScriptCode{Exemplo 11\newline%
Arquivo: \texttt{/js/exemplo11.js}}{projetos/capitulo07/ExemplosEmJavaScript/web/js/exemplo11.js}



\subsection{AJAX jQuery e com Fetch API Processando JSON}

\url{https://developer.mozilla.org/en-US/docs/Web/API/Fetch_API/Using_Fetch}

Listagem~\thechapter.\ref{listagem:projetos/capitulo07/ExemplosEmJavaScript/web/js/exemplo12.js}

\javaScriptCode{Exemplo 12\newline%
Arquivo: \texttt{/js/exemplo12.js}}{projetos/capitulo07/ExemplosEmJavaScript/web/js/exemplo12.js}



\section{Resumo}


\section{Exercícios}


\section{Projetos}
