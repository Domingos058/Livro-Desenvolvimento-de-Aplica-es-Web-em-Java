\chapter{JavaScript}\label{cap:javaScript}
\epigraph{``\textit{A vida vai ficando cada vez mais dura perto do topo}''.}{Friedrich Nietzsche}

\lettrine[lines=4, lhang=0.1, lraise=0, loversize=0.2, findent=0.1em]{\textcolor{corAzulTema}{N}}{ESTE} Capítulo temos com objetivo aprender as construções básicas da linguagem de programação JavaScript, vastamente utilizada no desenvolvimento de aplicações Web.

\section{Introdução}

Chegamos a talvez à cereja do bolo, ou à cereja do livro ou então à cereja do desenvolvimento para Web: a linguagem de \textit{script} JavaScript. A primeira coisa que precisamos deixar claro é que Java e JavaScript são duas linguagens diferentes, com sintaxe baseada em C e com construções similares, mas o comum entre as duas para por aqui. A linguagem JavaScript não é uma linguagem orientada a objetos, mas sim baseada em protótipos. Nesse tipo de linguagem não existem classes, mas apenas objetos. Novos objetos são criados a partir de cópias de objetos existentes. O JavaScript moderno, baseado no padrão ECMAScript, possui algumas construções que lembram a orientação a objetos como classes e herança de entre classes, mas tudo isso é \textit{syntax sugar} para simplificar coisas que já existiam anteriormente. Outras construções também são suportadas na linguagem como as funções de primeira classe etc.

Neste Capítulo não entraremos em detalhes sobre a história da linguagem e da sua evolução, mas sim no que eu acho útil e fundamental para que possamos começar a usar a linguagem. Iremos ver uma visão geral da linguagem como declaração de variáveis, manipulação do \textit{Document Object Model} (DOM) e requisições assíncronas. Durante o Capítulo serão apresentadas inúmeras caixas do tipo ``Saiba Mais'' com \textit{links} úteis. A maioria desses links serão da \textit{Mozilla Developer Network}\footnote{\url{https://developer.mozilla.org/}} (MDN), uma referência confiável e oficial da maioria, senão de todas, das tecnologias para Web. Sempre fornecerei os \textit{links} da versão em inglês do site, mas se quiser, você pode verificar a versão em português clicando no botão ``\textit{Change language}'' presente em todas as páginas do site. Sinceramente recomendo a leitura em inglês, pois o texto sempre estará completo, atualizado com a terminologia correta.

\begin{saibaMais}
    Quer conhecer um pouco mais da história e dos detalhes da linguagem JavaScript? Veja o \textit{link} \url{https://developer.mozilla.org/en-US/docs/Web/JavaScript}.
\end{saibaMais}

\begin{saibaMais}
    A referência da linguagem JavaScript pode ser acessada pelo \textit{link} \url{https://developer.mozilla.org/en-US/docs/Web/JavaScript/Reference}.
\end{saibaMais}

\begin{saibaMais}
    As documentações e referências das tecnologias e APIs usadas para o desenvolvimento para Web podem ser acessadas pelo \textit{link} \url{https://developer.mozilla.org/en-US/docs/Web/API}.
\end{saibaMais}

\begin{saibaMais}
    Caso deseje fazer um tutorial completo sobre a linguagem recomendo o tutorial da própria MDN, que é muito bom: \url{https://developer.mozilla.org/en-US/docs/Learn/JavaScript}.
\end{saibaMais}


Antes de começarmos a falar do JavaScript propriamente dito, vamos montar nosso palco, que é um projeto Java para Web. Neste Capítulo ainda faremos a construção dos projetos do zero, mas a partir do próximo focarei apenas nas novidades que serão apresentadas.

Vamos lá. Crie um projeto Java Web com o nome de ``ExemplosEmJavaScript'' da forma que tem feito até aqui. Os passos descritos a seguir são somente estruturais. Os respectivos códigos dos arquivos serão apresentados e explicados posteriormente. Sendo assim, no nó \destaque{\textit{Web Pages}} do projeto:

\begin{itemize}
    \item Remova o arquivo \texttt{index.html};
    \item Crie um JSP chamado \texttt{index.jsp};
    \item Crie uma pasta chamada \texttt{css};
    \item Crie uma pasta chamada \texttt{js} (JavaScript);
    \item Dentro da pasta \texttt{css} crie um arquivo CSS chamado \texttt{estilos.css};
    \item Dentro da pasta \texttt{js} crie 12 arquivos JavaScript, chamados \texttt{exemplo01.js}, \texttt{exemplo02.js}... \texttt{exemplo12.js};
    \item Em \destaque{\textit{Source Packages}} crie os pacotes \texttt{exemplosemjavascript.pojo} e\newline%
    \texttt{exemplosemjavascript.servlets};
    \item No pacote \texttt{exemplosemjavascript.pojo} crie uma classe chamada \texttt{Pessoa};
    \item No pacote \texttt{exemplosemjavascript.servlets} crie os Servlets \texttt{CalculaTabuadaServlet} e \texttt{ListagemPessoasServlet};
    \item Importe e insira no projeto a biblioteca \texttt{Jakarta EE Web 8 API};
    \item Clique com o botão direito do mouse no nó raiz do projeto e escolha o último item do menu de contexto, chamado \destaque{\textit{Properties}};
    \begin{itemize}
        \item Do lado esquerdo, em \destaque{\textit{Categories:}}, clique no item \destaque{CDNJS}, situado dentro do nó \destaque{\textit{JavaScript Libraries}};
        \item Do lado direito, clique no botão \destaque{\textit{Add}};
        \item No diálogo que abriu, intitulado \destaque{\textit{Add CDNJS Library}}, preencha o campo \destaque{\textit{Find:}} com ``\texttt{jquery}'' (sem as aspas) e clique em \destaque{\textit{Search}};
        \item Após a busca na \textit{Content Delivery Network} (CDN) a aparecerão diversos componentes na \textit{Graphical User Interface} (GUI). Em \destaque{\textit{Libraries:}} escolha \texttt{jquery}, provavelmente o primeiro item;
        \item Em \destaque{\textit{Files:}} marque a \textit{checkbox} na frente do item \texttt{jquery.min.js} e clique em \destaque{\textit{Add Library}} como na Figura~\ref{fig:cap07AddCDNJSLibrary};
        \FloatBarrier
        \begin{figure}[!htbp]
            \centering
            \caption{Adicionando uma biblioteca JavaScript}
            \includegraphics[scale=0.7]{imagens/cap07AddCDNJSLibrary}
            \\\textbf{Fonte:} Elaborada pelo autor
            \label{fig:cap07AddCDNJSLibrary}
        \end{figure}
        \FloatBarrier
        \item Clique em \destaque{OK}. A biblioteca jQuery será baixada e inserida no projeto dentro da pasta \texttt{js/libs/jquery}.
    \end{itemize}
\end{itemize}

Realizando todos os passos descritos anteriormente, você terá um projeto com a estrutura apresentada na Figura~\ref{fig:cap07EstruturaDoProjeto}. Agora vamos começar a preencher cada um dos arquivos e aprender o que está acontecendo em cada um deles.

\FloatBarrier
\begin{figure}[!htbp]
    \centering
    \caption{Estrutura do projeto}
    \includegraphics[scale=0.7]{imagens/cap07EstruturaDoProjeto}
    \\\textbf{Fonte:} Elaborada pelo autor
    \label{fig:cap07EstruturaDoProjeto}
\end{figure}
\FloatBarrier

Começaremos com o \texttt{index.jsp} apresentado na Listagem~\thechapter.\ref{listagem:projetos/capitulo07/ExemplosEmJavaScript/web/index.jsp}. Entre as linhas 10 e 22 usamos a \textit{tag} \inlineHTMLCode{<script>} para carregarmos no documento treze arquivos de código JavaScript. Inclusive, essa mesma \textit{tag} pode ser utilizada para inserir código JavaScript no próprio documento. Veremos isso mais adiante. Na linha 10 é carregada a biblioteca jQuery que há alguns anos atrás era absolutamente relevante, mas que hoje em dia tem caído em desuso visto a evolução do JavaScript. Ela será tratada no livro pela sua importância em software legado, por facilitar e padronizar algumas coisas e também, é claro, por preferência minha :D. No restante das linhas são associados os arquivos com os exemplos que aprenderemos. O restante do documento consiste na construção de uma interface com alguns componentes e \textit{tags} que serão manipulados pelo nosso código em JavaScript. 

Os cinco primeiros exemplos são relativos às construções principais da linguagem. Veja que da linha 30 à 34 temos um parágrafo (\textit{tag} \inlineHTMLCode{<p>}) com um botão dentro (\textit{tag} \inlineHTMLCode{<button>}) e que em seu evento \textit{click}, representado pelo atributo \texttt{onclick}, é registrado uma função tratadora/manipuladora/ouvinte de evento (\textit{event handler} ou \textit{event listener}). Para registrar uma função como tratadora de um determinado evento, basta inserir seu nome no valor do atributo \texttt{onclick} e adicionar, entre os parânteses da mesma, a palavra \texttt{event}. Esse \texttt{event} carregará o objeto do evento que será disparado pela \textit{tag} e ouvido pela função. Essa função precisa estar declarada e implementada em algum lugar. No nosso caso, estará no arquivo \texttt{exemplo01.js}, referenciado acima. Note que como o JavaScript é interpretado, para se poder usar algo, essa ``coisa'' precisa ter sido declarada antes ou as vezes ``vista'' pelo interpretador pela primeira vez, sendo que esse segundo comportamento pode gerar muitos problemas caso não seja entendido apropriadamente, mas veremos isso também. Resumindo, ao se clicar (\texttt{onclick}) nesse primeiro botão, afunção \inlineJavaScriptCode{executarExemplo01(event)} será invocada. Fácil não é? Existe uma infinidade de eventos permitidos para cada \textit{tag}, mas falaremos de alguns deles à medida que for necessário. CEsse padrão de um botão invocando uma função se repetirá em praticamente todos os exemplos. Os cinco primeiros têm a mesma estrutura.

Nos exemplos 06, 07 e 08 trataremos da manipulação das \textit{tags}, como dinamicamente inserir conteúdo nas mesmas ou ler/escrever dados em componentes de formulário. Esses exemplos estão dentro da seção ``Manipulação do DOM'', onde DOM significa \textit{Document Object Mode}l que nos bastidores é uma árvore composta de objetos que representam o resultado do \textit{parse} do arquivo HTML pelo navegador ou cliente. Usando JavaScript podemos mexer nessa árvore, alterando atributos dos nós, que na maioria das vezes representam as \textit{tags}, além de inserir e remover nós. Todas as modificações são replicadas automaticamente pelo navegador que entende que a árvore foi alterada e precisa ser redesenhada no processo de renderização do documento. Antigamente, quando isso era novidade há mais de 20 anos atrás, era chamado de ``HTML Dinâmico'' (\textit{Dynamic} HTML (DHTML)). Sim, estou ficando velho :D. Perceba que nesses exemplos, além dos botões temos \textit{divs} que serão usadas para mostrar o resultado de algum processamento, além de componentes de formulário e outros botões no exemplo 08.

No exemplo 09 falaremos um pouco de tratamento de eventos, como já comentei anteriormente. Nesse exemplo, na linha 171, temos a invocação da função\newline%
\inlineJavaScriptCode{registrarEventosExemplo09()} em que, programaticamente, faremos o registro dos ouvintes de eventos ao invés de usar os atributos prefixados com ``\texttt{on}'' das \textit{tags}.

No exemplo 10 trataremos do uso da \textit{tag} \inlineHTMLCode{<canvas>} usada para desenhar programaticamente, realizando uma simulação física de uma bolinha.

Nos dois últimos exemplos trataremos das requisições assíncronas e de intercâmbio de dados entre cliente e servidor.

\htmlCode{Página principal da aplicação\newline%
Arquivo: \texttt{/index.jsp}}{projetos/capitulo07/ExemplosEmJavaScript/web/index.jsp}

Caso queira, durante os testes de execução, comente trechos do código do \texttt{index.jsp} para que você não precise ficar rolando a página para chegar em alguma parte toda vez que for testar uma funcionalidade.

Na Listagem~\thechapter.\ref{listagem:projetos/capitulo07/ExemplosEmJavaScript/web/css/estilos.css} é apresentado o arquivo com as folhas de estilo usadas no \texttt{index.jsp}. O código já contém os comentários para você entender o que se trata cada coisa.

\cssCode{Folhas de estilo do projeto\newline%
Arquivo: \texttt{/index.jsp}}{projetos/capitulo07/ExemplosEmJavaScript/web/css/estilos.css}

\begin{saibaMais}
    Sobre CSS, consulte \url{https://developer.mozilla.org/en-US/docs/Web/CSS} e \url{https://developer.mozilla.org/en-US/docs/Web/CSS/Reference}.
\end{saibaMais}

Nas próximas três listagens serão mostrados os componentes do lado do servidor que utilizaremos para os dois últimos exemplos. Na Listagem~\thechapter.\ref{listagem:projetos/capitulo07/ExemplosEmJavaScript/src/java/exemplosemjavascript/pojo/Pessoa.java} definimos a classe \texttt{Pessoa}, um \textit{Plain Old Java Object} (POJO) ou \textit{Value Object} (VO) que é uma classe que utilizaremos para criar objetos para transportar dados.

\javaCode{Classe Pessoa\newline%
Arquivo: \texttt{exemplosemjavascript/pojo/Pessoa.java}}{projetos/capitulo07/ExemplosEmJavaScript/src/java/exemplosemjavascript/pojo/Pessoa.java}

Na Listagem~\thechapter.\ref{listagem:projetos/capitulo07/ExemplosEmJavaScript/src/java/exemplosemjavascript/servlets/CalculaTabuadaServlet.java} é apresentado o código do Servlet \texttt{CalculaTabuadaServlet}, mapeado em \texttt{/calcularTabuada} que recebe um valor inteiro e retorna o texto representando a ``tabuada'' do número processado. Esse retorno é gerado pelo próprio Servlet no seu fluxo de saída (linhas 41, 42 e 43), sendo que o objeto response é configurado para indicar ao cliente que o que está chegando é no formato de texto puro (linha 28).

\javaCode{Servlet de tabuada\newline%
Arquivo: \texttt{exemplosemjavascript/servlets/CalculaTabuadaServlet.java}}{projetos/capitulo07/ExemplosEmJavaScript/src/java/exemplosemjavascript/servlets/CalculaTabuadaServlet.java}

Por fim, na Listagem~\thechapter.\ref{listagem:projetos/capitulo07/ExemplosEmJavaScript/src/java/exemplosemjavascript/servlets/ListagemPessoasServlet.java} é apresentado o código do Servlet \texttt{ListagemPessoasServlet}, mapeado em \texttt{/listarPessoas} que indica ao cliente que os dados que serão retornados irão no formato JavaScript Object Notation\footnote{O formato JSON será tratado no exemplo 05. Por enquanto assuma que é uma forma de codificar os dados de um objeto em forma de texto.} (JSON) (linha 33). Nesse Servlet é criada uma lista de objetos do tipo \texttt{Pessoa}, baseada na quantidade recebida via requisição e essa lista é serializada em JSON usando a camada de \textit{bindind} JSON-B do Java/Jakarta EE. Na linha 35 é criado o objeto serializador e na linha 56 ele é usado, convertendo a lista com os objetos do tipo \texttt{Pessoa} em uma representação em texto, que no nosso caso é o JSON.

\javaCode{Servlet de listagem de pessoas usando JSON\newline%
Arquivo: \texttt{exemplosemjavascript/servlets/ListagemPessoasServlet.java}}{projetos/capitulo07/ExemplosEmJavaScript/src/java/exemplosemjavascript/servlets/ListagemPessoasServlet.java}

Agora que temos toda a infraestrutura básica do nosso projeto, podemos começar a falar sobre JavaScript. Vamos começar!



\section{Funções de E/S e Operadores Aritméticos}

Começaremos nossa breve jornada de descoberta da linguagem JavaScript aprendendo uma forma de obter dados do usuário, que normalmente não é usada em um software em produção, mas para aprender conceitos vai nos servir no momento, como gerar saída, declarar variáveis e realizar as operações aritméticas básicas. Na Listagem~\thechapter.\ref{listagem:projetos/capitulo07/ExemplosEmJavaScript/web/js/exemplo01.js} pode ser visto o código completo do primeiro exemplo. Veja que logo na primeira linha há a declaração da função \inlineJavaScriptCode{executarExemplo01(event)} que é a função que tratará o evento \texttt{click} do primeiro botão do \texttt{index.jsp}.

\javaScriptCode{Exemplo 01\newline%
Arquivo: \texttt{/js/exemplo01.js}}{projetos/capitulo07/ExemplosEmJavaScript/web/js/exemplo01.js}

Na linha 5 é declarada uma variável local usando a palavra chave \inlineJavaScriptCode{let}\footnote{Veremos o propósito da palavra chave \texttt{let} no exemplo 02.}, com o identificador \inlineJavaScriptCode{n1} e atribuímos a ela o retorno da função \inlineJavaScriptCode{prompt}. Essa função recebe como parâmetro uma String que, ao ser executada, apresenta ao usuário um diálogo com uma mensagem -vinda da String passada- um campo de texto, um botão de confirmação e um de cancelamento. Ao se clicar no botão de confirmação o valor fornecido do campo de texto será retornado ao chamador, no caso, atribuído à variável \inlineJavaScriptCode{n1} e se o diálogo for cancelado, será retornado o valor \inlineJavaScriptCode{null}. O retorno é do tipo String. Note que não declaramos o tipo das variáveis em JavaScript, pois a tipagem das variáveis é dinâmica, visto que o tipo de cada variável depende do valor atribuído ou referenciado por ela.

Na linha 9 fazemos basicamente a mesma coisa para a variável \inlineJavaScriptCode{n2}, mas o retorno da função \inlineJavaScriptCode{prompt} é usada como argumento da função \inlineJavaScriptCode{Number} que converterá a String retornada por \inlineJavaScriptCode{prompt} em um número e então esse valor será atribuído a \inlineJavaScriptCode{n2}.

Entre as linhas 12 e 16 declaramos cinco novas variáveis e atribuímos a elas o resultado de cinco operações. Note que como \inlineJavaScriptCode{n1} referencia uma String, o operador \inlineJavaScriptCode{+} será tratado como operador de concatenação de Strings ao invés de adição, ou seja, \inlineJavaScriptCode{n2} será convertida para String e concatenada com \inlineJavaScriptCode{n1}! Os outros operadores como são aplicáveis apenas à números, \inlineJavaScriptCode{n1} será convertido implicitamente e a operação será realizada. O resultado disso será visto na saída que será gerada e exibida.

Falando da saída, na linha 19 declaramos a variável \inlineJavaScriptCode{saida} e concatenamos diversas Strings para gerar o resultado. Em JavaScript existem três literais para Strings:
  
\begin{enumerate}
    \item Delimitadas por aspas simples (apóstrofo): \inlineJavaScriptCode{'uma string'};
    \item Delimitadas por aspas duplas (aspas): \inlineJavaScriptCode{"outra string"};
    \item Delimitadas por acento grave (crase): \inlineJavaScriptCode{`mais uma string`};
\end{enumerate}

Os dois primeiros são análogos, com a diferença que quando se usa aspas simples como delimitador e queremos ter uma aspas simples dentro da String, precisamos escapá-la com contrabarra (barra invertida) e as aspas duplas não precisam. Por exemplo, \inlineJavaScriptCode{'a\'b"c'} corresponde à \destaque{\texttt{a\textquotesingle{}b"c}}. Quando delimitamos a String com aspas duplas temos o contrário, ou seja, \inlineJavaScriptCode{"a'b\"c"} correspondendo à \destaque{\texttt{a\textquotesingle{}b"c}}.

O terceiro tipo de delimitador é mais interessante, pois permite que façamos a interpolação de valores dentro da String usando uma notação parecida com a da EL do Java Web, mas que não tem relação a não ser a sintaxe similar. Para o nosso exemplo, se \inlineJavaScriptCode{n1} valer \inlineJavaScriptCode{"10"} e \inlineJavaScriptCode{n2} valer \inlineJavaScriptCode{5}, o resultado de \inlineJavaScriptCode{`${n1} e ${n2}`} será \destaque{\texttt{10 e 5}}.

Por fim, para apresentar a String gerada, usamos duas formas. A primeira, na linha 29, com a função \inlineJavaScriptCode{alert} que, assim como \inlineJavaScriptCode{prompt}, é bloqueante, fazendo a execução do código parar naquele ponto ao esperar a interação do usuário. Essa função recebe uma String como parâmetro e a mostra num diálogo ao ser executada. A outra forma é usando a função \inlineJavaScriptCode{log} do objeto \inlineJavaScriptCode{console}, que recebe um objeto como parâmetro e o mostra no console do navegador. No nosso exemplo, a exibição no console está condicionada ao retorno da função \inlineJavaScriptCode{confirm} que exibe uma mensagem ao usuário e aguarda a interação. Caso o usuário confirme a mensagem, a função retornará um valor verdadeiro, usado na estrutura condicional \inlineJavaScriptCode{if} do exemplo.



\section{Declarações de Variáveis e Suas Implicações}

Nesta seção trataremos as variáveis e as declarações delas em JavaScript. Como já dito, as variáveis não tem um tipo definido, visto que a linguagem é dinamicamente tipada, implicando que o tipo da variável varia de acordo com o que ela referencia. Em JavaScript temos Strings, números, valores lógicos, funções, objetos entre outros.

Toda variável em JavaScript ao ser declarada passará pelo processo de \textit{hoisting}. Nesse processo, a variável será elevada ou içada até o início ou topo do contexto em que ela foi declarada e que passará a existir. A ideia é que quando o interpretador encontra uma declaração de variável e ela é bem sucedida, ou seja, é sintática e semanticamente correta, ela passará a existir como se houvesse sido declarada no início do escopo em que reside.

Podemos influenciar em como esse içamento ocorrerá em relação à inicialização das variáveis. Veja a lista abaixo, temos quatro formas de declarar variáveis:

\begin{enumerate}
    \item \inlineJavaScriptCode{let variavel = "valor";};
    \item \inlineJavaScriptCode{const constante = "valor";};
    \item \inlineJavaScriptCode{var variavel = "valor";};
    \item \inlineJavaScriptCode{variavel = "valor";};
\end{enumerate}

Quando a palavra-chave \inlineJavaScriptCode{let} é usada, a variável só poderá ser usada depois da sua inicialização, mesmo havendo \textit{hoisting} para sua declaração. O mesmo acontece com as constantes, declaradas com \inlineJavaScriptCode{const}. Já as variáveis declaradas com a palavra-chave \inlineJavaScriptCode{var} serão inicializadas com \inlineJavaScriptCode{undefined}. Por fim, as variáveis que são declaradas sem indicar nenhuma dessas três palavras-chave passarão a existir no escopo global, o que pode trazer uma série de problemas. Imagine que você declarou mais de uma variável com o mesmo nome em dois ou mais escopos diferentes. A declaração de fato ocorrerá quando o interpretador a encontrar pela primeira vez e, independende de onde for, ela passará a existir no escopo global e a partir desse ponto você pode perder o controle do valor que a variável referencia se não tomar muito cuidado com o que está fazendo. O ideal é não utilizar ok?

O exemplo apresentado na Listagem~\thechapter.\ref{listagem:projetos/capitulo07/ExemplosEmJavaScript/web/js/exemplo02.js} mostra todos esses efeitos quando for executado. O ``problema'' da variável declarada sem \inlineJavaScriptCode{let}, \inlineJavaScriptCode{var} ou \inlineJavaScriptCode{const} pode ser reproduzido ao se clicar pela segunda vez no botão do exemplo 02.

\javaScriptCode{Exemplo 02\newline%
Arquivo: \texttt{/js/exemplo02.js}}{projetos/capitulo07/ExemplosEmJavaScript/web/js/exemplo02.js}

Veja o exemplo, todo o código está comentado, não sendo necessário entrar em mais detalhes. Recomendo que você dê uma olhada nos \textit{links} disponibilizados nas próximas duas caixas ``Saiba Mais''.

\begin{saibaMais}
    Para uma explicação mais detalhada sobre essas implicações, acesse \url{http://www.constletvar.com/}.
\end{saibaMais}

\begin{saibaMais}
    Para mais detalhes sobre variáveis de clarações em JavaScript, acesse \url{https://developer.mozilla.org/en-US/docs/Web/JavaScript/Reference/Statements}.
\end{saibaMais}

Esse conceito pode gerar muita confusão, inclusive se declararmos uma variável com \inlineJavaScriptCode{var} no contexto global (fora de funções) ela será uma variável global (uma propriedade do objeto \texttt{window}), ao passo que dentro de uma função ela terá escopo local, assim como \inlineJavaScriptCode{let} e \inlineJavaScriptCode{const}, mas inicializada como \inlineJavaScriptCode{undefined}. Ainda, é importante frisar que uma constante tem ligação imutável (\textit{immutable binding}) com o que ela referencia, ou seja, ela não pode receber um novo valor, mas o valor que ela referencia pode ser modificado (ela não é imutável), por exemplo, se for um objeto e quisermos alterar alguma de suas propriedades.



\section{Estruturas Condicionais e Operadores}

Em JavaScript temos as mesmas estruturas condicionais presentes na maioria das linguagens de programação ou seja, um \inlineJavaScriptCode{if} com \inlineJavaScriptCode{else} aninhados e opcionais e um \inlineJavaScriptCode{switch}. Os operadores relacionais e lógicos também são os operadores padrão encontrados na maioria das linguages derivadas de C, com a adição de mais dois operadores relacionais que são o operador de identidade (\inlineJavaScriptCode{===}) e o operador de não identidade (\inlineJavaScriptCode{!==}). Ao passo que os operadores de igualdade e de desigualdade verificam se o valor dos operandos comparados são respectivamente iguais ou difentes, inclusive após a conversão implícita de algum deles, os operadores de identidade e de não identidade verificam, além do valor (sem conversão implícita), respectivamente, se o tipo é o mesmo ou diferente. Na Listagem~\thechapter.\ref{listagem:projetos/capitulo07/ExemplosEmJavaScript/web/js/exemplo03.js} pode ser visto o emprego das estruturas condicionais e a declaração de variáveis com alguns valores permitidos.

\javaScriptCode{Exemplo 03\newline%
Arquivo: \texttt{/js/exemplo03.js}}{projetos/capitulo07/ExemplosEmJavaScript/web/js/exemplo03.js}

\begin{saibaMais}
    Para mais detalhes sobre os operadores em JavaScript, acesse \url{https://developer.mozilla.org/en-US/docs/Web/JavaScript/Reference/Operators}.
\end{saibaMais}


\section{Estruturas de Repetição e Arrays}

Os Arrays em JavaScript atuam como arrays na linguagem Java e C, sendo indexados iniciando em 0, mas podendo crescer ou diminuir quando necessário, assemelhando-se mais com listas do que arrays de tamanho fixo. Podemos usar a notação de colchetes para simular arrays associativos (tabelas de símbolos), mas de fato o que acontece é que estamos criando ou lendo propriedades do objeto do array. Na Listagem~\thechapter.\ref{listagem:projetos/capitulo07/ExemplosEmJavaScript/web/js/exemplo04.js} pode se ver a criação de três arrays e a utilização da estrutura de repetição \inlineJavaScriptCode{for} para iterar por seus elementos.

\javaScriptCode{Exemplo 04\newline%
Arquivo: \texttt{/js/exemplo04.js}}{projetos/capitulo07/ExemplosEmJavaScript/web/js/exemplo04.js}

Na linha 4 o um array de uma dimensão é criado e inicializado com os valores 1, 2, 3 e 4, contidos respectivamente nas posições 0, 1, 2 e 3. Na linha 7 é criado um array em que nas posições 0 e 1 estão contidos dois outros arrays, um com os valores 1 e 2 e o outro com os valores 3 e 4.

Na linha 10 um novo array vazio é criado e entre as linhas 11 e 14 são inseridas quatro propriedades no objeto em si. Note que essas propriedades são criadas no objeto e pela sintaxe usada, há a impressão que estamos lidando com o array como se fosse um array associativo ou uma tabela de símbolos, mapa ou dicionário, mas não é o caso! Podemos usar a notação de colchetes para lidar com objetos para, por exemplo, acessar propriedades ou atributos que contenham espaço nos nomes. Note que posteriomente, ao se tentar iterar por esse array usando um \inlineJavaScriptCode{for} ``normal'', não se entrará no bloco da estrutura de repetição, visto que, apesar de parecer que o array contém quatro elementos, na verdade ele não tem nenhum, fazendo com que o atributo \inlineJavaScriptCode{length} valha zero.

A inserção e remoção de elementos dos arrays podem ser feitas usando os métodos \inlineJavaScriptCode{push} que insere um elemento no fim do array, \inlineJavaScriptCode{pop} que remove um elemento do fim, \inlineJavaScriptCode{unshift} que insere um elemento no início, \inlineJavaScriptCode{shift} que remove do início e \inlineJavaScriptCode{splice} que remove de uma posição arbitrária. Na caixa ``Saiba Mais'' abaixo você pode dar uma olhada na documentação do objeto Array da linguagem JavaScript.

\begin{saibaMais}
    Para mais detalhes sobre o tipo Array em JavaScript, acesse \url{https://developer.mozilla.org/en-US/docs/Web/JavaScript/Reference/Global_Objects/Array}.
\end{saibaMais}

Entre as linhas 24 e 26 usa-se um \inlineJavaScriptCode{for} para iterar pelos elementos do array \inlineJavaScriptCode{a1}. Entre as linhas 30 e 32 usa-se o método \inlineJavaScriptCode{forEach} do objeto Array para iterar pelos elementos, usando uma função de \textit{callback} para processar cada posição. A mesma coisa acontece entre as linhas 34 e 41, com a diferença de se usar a notação de closure para a definição da função de \textit{callback} utilizada no método \inlineJavaScriptCode{forEach} de \inlineJavaScriptCode{a2}.

Como mencionado anteriormente, o uso de um \inlineJavaScriptCode{for} padrão -ou mesmo um \inlineJavaScriptCode{forEach}- para \inlineJavaScriptCode{a3} não surtirá efeito, pois esse array está vazio! Nós inserimos quatro propriedades no mesmo, não quatro elementos. Quando usamos um array dessa forma, inclusive poderia ser qualquer objeto, e queremos iterar por essas propriedades temos basicamente duas formas: ou fazemos como entre as linhas 29 e 51, usando um \inlineJavaScriptCode{for ... in} ou obtemos as chaves do objeto como um array e as processamos usando um \inlineJavaScriptCode{forEach} como mostrado entre as linhas 54 e 56.

Além do método \inlineJavaScriptCode{forEach} existem alguns outros para o processamento dos elementos de um array. Dois desses métodos são mostrados a partir da linha 59: \inlineJavaScriptCode{every} e \inlineJavaScriptCode{some}. Como os nomes sugerem, o primeiro é utilizado com a premissa de testar alguma condição em todos os elementos do array, enquanto o outro espera-se que algum elemento não se enquadre em algo desejado. Podemos utilizar esses métodos para, por exemplo, executar uma busca/pesquisa sequencial/linear no array e parar a iteração quando o elemento for encontrado, retornando um valor verdadeiro ou falso, dependendo da situação e do método empregado. Veja os comentários do exemplo e teste a execução do código do exemplo para entender exatamente do que se trata.



\section{``Classes'', Objetos e JSON}

Antes do ECMAScript 2015 (sexta edição) a criação de objetos com um ``tipo'' era feito a partir do uso de uma função e o operador \inlineJavaScriptCode{new}. A partir do ECMAScript 2015 existe uma sintaxe para a definição de tipos abstratos de dados em forma de classes, inclusive permitindo herança entre tipos definidos. Na Listagem~\thechapter.\ref{listagem:projetos/capitulo07/ExemplosEmJavaScript/web/js/exemplo05.js}, entre as linhas 1 e 17 é definida a classe \texttt{Forma}. As classes em JavaScript podem ter apenas um construtor e, dentro dele, as propriedades do objeto devem ser criadas utilizando a palavra chave \inlineJavaScriptCode{this}.

\javaScriptCode{Exemplo 05\newline%
Arquivo: \texttt{/js/exemplo05.js}}{projetos/capitulo07/ExemplosEmJavaScript/web/js/exemplo05.js}

Entre as linhas 20 e 43 cria-se as classes \texttt{Retangulo} e \texttt{Circulo} que herdam de \texttt{Forma}, sobrescrevendo o método \inlineJavaScriptCode{calcularArea()}. Nas linhas 47 e 48 cria-se respectivamente uma instância de \texttt{Retangulo} e uma de \texttt{Circulo}. Entre as linhas e 66 cria-se um novo objeto genético, usando o inicializador de objetos (abre e fecha chaves). Note que esse objeto e os outros dois tem como tipo \texttt{object}, mas é possível verificar se são instância de uma determinada classe/tipo usando o operador \inlineJavaScriptCode{instanceof} como visto entre as linhas 69 e 87.

Podemos também representar objetos inteiros como Strings usando a notação de objetos JavaScript, chamada de JSON. Veja na linha 91 onde temos uma String codificando um objeto usando algo simular à notação de inicilização de objetos. Isso é o JSON. Para transformar essa String em um objeto, usa-se o método \inlineJavaScriptCode{parse()} do objeto global \inlineJavaScriptCode{JSON} e, quando se quer o contrário, ou seja, transformar um objeto em uma String no formato JSON, usa-se o método \inlineJavaScriptCode{stringfy()} do objeto \inlineJavaScriptCode{JSON}. O formato JSON é amplamanete utilizado atualmente em detrimento do formado XML, pois é mais enxuto, ou seja, é necessário menos texto para codificar o mesmo objeto em JSON do que em XML. Tanto o formato JSON quanto XML são utilizados para a serialização de objetos, ou seja, transformar uma notação binária, específica de linguagem, em uma representação serial fácil de ser processada e independente de plataforma. O processo de converter um objeto para um formato de intercâmbio de dados é chamado de serialização, enquanto o processo inverso, ou seja, a partir de um formato de intercâmbio de dados gerar um objeto específico de tecnologia é chamado de desserialização.

O restante do código do exemplo é facilmente entendido ao ser lido. Nas próximas caixas ``Saiba Mais'' há varios links úteis sobre o tema.

\begin{saibaMais}
    Para saber mais sobre classes em JavaScript, acesse \url{https://developer.mozilla.org/en-US/docs/Web/JavaScript/Reference/Classes}.
\end{saibaMais}

\begin{saibaMais}
    Para consultar a documentação sobre o tipo String em JavaScript, acesse \url{https://developer.mozilla.org/en-US/docs/Web/JavaScript/Reference/Global_Objects/String}.
\end{saibaMais}

\begin{saibaMais}
    Para consultar a documentação sobre JSON em JavaScript, acesse \url{https://developer.mozilla.org/en-US/docs/Web/JavaScript/Reference/Global_Objects/JSON}.
\end{saibaMais}



\section{Manipulação do DOM}

Nesta Seção aprenderemos a manipular o DOM com objetivo de extrair dados do documento ou então modificá-lo em tempo de execução. Isso já foi comentado brevemente anteriormente!

\begin{saibaMais}
    A documentação do DOM pode ser vista no \textit{link} \url{https://developer.mozilla.org/en-US/docs/Web/API/Document_Object_Model}.
\end{saibaMais}

Vamos manipular o DOM de duas formas, uma usando JavaScript puro, que normalmente é mais verboso, e a outra usando a biblioteca jQuery\footnote{\url{https://jquery.com/}}, que já foi padrão na construção de aplicações para Web e tem caído em desuso, mas ainda é relevante, principalmente por facilitar algumas coisas e ainda estar presente de forma consistente em diversas aplicações criadas e que você provavelmente terá que dar manutenção na sua vida profissional.


\subsection{JavaScript Puro}

A ideia desse exemplo é que ao se clicar no botão, um novo nó da \textit{tag} \inlineHTMLCode{<p>} seja inserido na \inlineHTMLCode{<div>} que tem \texttt{id} igual à \texttt{divExemplo06}, usando um contador para mostrar as sucessivas inserções. Na linha 6 da Listagem~\thechapter.\ref{listagem:projetos/capitulo07/ExemplosEmJavaScript/web/js/exemplo06.js} obtém-se tal \inlineHTMLCode{<div>} utilizando o método \inlineJavaScriptCode{getElementById("id")} do objeto \inlineJavaScriptCode{document}. Caso bem-sucedido, o método retornará uma referência ao nó dessa \inlineHTMLCode{<div>} no DOM e então poderemos manipulá-lo! Na linha 9 criamos um novo nó para inserir na \inlineHTMLCode{<div>} do tipo correspondente à \textit{tag} \inlineHTMLCode{<p>}. Entre as linhas 12 e 13 inserimos o conteúdo do parágrafo criado e defimos qual é sua classe. Veja no arquivo \texttt{estilos.css} que temos a definição de um seletor de classe chamado \inlineCSSCode{.pDOM} que reflitirá na inserção desses parágrafos, sendo que todo item par terá uma cor de fundo diferente. Por fim, na linha 16 esse novo parágrafo é inserido na \inlineHTMLCode{<div>} e essa alteração é prontamente refletida na renderização do documento! Pare e pense agora todas as possibilidades existentes!

\javaScriptCode{Exemplo 06\newline%
Arquivo: \texttt{/js/exemplo06.js}}{projetos/capitulo07/ExemplosEmJavaScript/web/js/exemplo06.js}


\subsection{Usando jQuery}

Talvez os dois principais diferenciais ou chamarizes da jQuery que a tornaram famosa é a utilização da sintaxe de seletores do CSS para obter elementos do DOM, o que já foi implementado de forma nativa no JavaScript e a facilidade para lidar com requisições assíncronas com o servidor, o que também já foi endereçado nas versões mais novas do JavaScript.

No exemplo apresentado na Listagem~\thechapter.\ref{listagem:projetos/capitulo07/ExemplosEmJavaScript/web/js/exemplo07.js} temos algo análogo ao exemplo anterior, só que usando as funcionalidades dessa biblioteca. Na linha 7 obtemos a \inlineHTMLCode{<div>} que tem \texttt{divExemplo07} como \texttt{id} usando a notação de seletores do CSS! As funcionalidades da jQuery são acessadas através do símbolo de cifrão (\texttt{\$}) que é um alias para o objeto chamado \texttt{jQuery}, disponível para ser utilizado nos script a partir do momento em que a biblioteca é inserida no documento. A criação do parágrafo também é mais direta, bastando inserir o par de \textit{tags} como String no argumento da função \texttt{\$} e então encadear a invocação dos métodos \inlineJavaScriptCode{html} e \inlineJavaScriptCode{addClass}. A linha 16 é idêntica ao exemplo anterior.

\javaScriptCode{Exemplo 07\newline%
Arquivo: \texttt{/js/exemplo07.js}}{projetos/capitulo07/ExemplosEmJavaScript/web/js/exemplo07.js}

\begin{saibaMais}
    Se quiser aprender um pouco mais sobre a jQuery, acesse \url{https://learn.jquery.com/}.
\end{saibaMais}



\section{Manipulação de Formulários}

A ideia central presente neste Capítulo é fazer com que você entenda um pouco sobre JavaScript para podermos ter formulários mais sofisticados, permitindo a construção de cadastros que envolvam relacionamos muitos-para-muitos. Claro que estamos vendo várias coisas a mais, mas a ideia é dar subsídios para implementarmos coisas novas no projeto iniciado no Capítulo~\ref{cap:primeiroProjeto}. No exemplo apresentado na Listagem~\thechapter.\ref{listagem:projetos/capitulo07/ExemplosEmJavaScript/web/js/exemplo08.js} veremos a obtenção e inserção de dados em componentes de formulários. Detalharei alguns pontos do código para não ficar muito maçante e o restante é com você. Tenho certeza que entenderá o que está acontecendo com base no que foi visto até agora.

\javaScriptCode{Exemplo 08\newline%
Arquivo: \texttt{/js/exemplo08.js}}{projetos/capitulo07/ExemplosEmJavaScript/web/js/exemplo08.js}

Na linha ....


\section{Eventos}

Texto.

Listagem~\thechapter.\ref{listagem:projetos/capitulo07/ExemplosEmJavaScript/web/js/exemplo09.js}

\javaScriptCode{Exemplo 09\newline%
Arquivo: \texttt{/js/exemplo09.js}}{projetos/capitulo07/ExemplosEmJavaScript/web/js/exemplo09.js}

\begin{saibaMais}
    A documentação completa sobre os eventos que podem ser tratados pode ser vista em  \url{https://developer.mozilla.org/en-US/docs/Web/Events}.
\end{saibaMais}



\section{Simulação Usando \textit{Canvas}}

Texto.

Listagem~\thechapter.\ref{listagem:projetos/capitulo07/ExemplosEmJavaScript/web/js/exemplo10.js}

\javaScriptCode{Exemplo 10\newline%
Arquivo: \texttt{/js/exemplo10.js}}{projetos/capitulo07/ExemplosEmJavaScript/web/js/exemplo10.js}

\begin{saibaMais}
    A API do Canvas pode ser vista em \url{https://developer.mozilla.org/en-US/docs/Web/API/Canvas_API}.
\end{saibaMais}



\section{Requisições Assíncronas e Intercâmbio de Dados}

Texto. Falar da Web Worker API

\begin{saibaMais}
    Mais sobre a API Web Workers \url{https://developer.mozilla.org/en-US/docs/Web/API/Web_Workers_APII}.
\end{saibaMais}


\subsection{AJAX com jQuery e com Fetch API}

Texto.

Listagem~\thechapter.\ref{listagem:projetos/capitulo07/ExemplosEmJavaScript/web/js/exemplo11.js}

\javaScriptCode{Exemplo 11\newline%
Arquivo: \texttt{/js/exemplo11.js}}{projetos/capitulo07/ExemplosEmJavaScript/web/js/exemplo11.js}

\begin{saibaMais}
    Documentação da função jQuery.ajax(): \url{https://api.jquery.com/jquery.ajax/}.
\end{saibaMais}

\begin{saibaMais}
    Documentação da Fetch API: \url{https://developer.mozilla.org/en-US/docs/Web/API/Fetch_API}.
\end{saibaMais}


\subsection{AJAX jQuery e com Fetch API Processando JSON}

Texto.

\url{https://developer.mozilla.org/en-US/docs/Web/API/Fetch_API/Using_Fetch}

Listagem~\thechapter.\ref{listagem:projetos/capitulo07/ExemplosEmJavaScript/web/js/exemplo12.js}

\javaScriptCode{Exemplo 12\newline%
Arquivo: \texttt{/js/exemplo12.js}}{projetos/capitulo07/ExemplosEmJavaScript/web/js/exemplo12.js}


\section{Resumo}


\section{Exercícios}


\section{Projetos}
