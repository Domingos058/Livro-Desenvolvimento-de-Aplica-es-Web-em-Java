\chapter*{Antes de começarmos...}
\addcontentsline{toc}{chapter}{Antes de começarmos...}
\chaptermark{Antes de começarmos...}

\lettrine[lines=4, lhang=0.1, lraise=0, loversize=0.2, findent=0.1em]{\textcolor{corTema}{T}}{UDO} que vimos até agora nos serviu principalmente para criar uma base sólida de como os componentes de uma aplicação Web em Java funcionam, além de termos visto algumas coisas importantes relacionadas à linguagem JavaScript, mas apesar de ser possível criar profissionalmente aplicações Web dessa forma, talvez você tenha notado o quão demorado alguns processos são, principalmente os relacionados à persistência dos dados das entidades e a definição dos controladores da aplicação.

Nos próximos Capítulos focaremos em situações mais próximas da realidade de um desenvolvedor de aplicações Web em Java, aprendendo o essencial de vários \textit{frameworks} e bibliotecas ao aplicar o que iremos aprender na reconstrução do ``Sistema de Venda de Produtos'' do Capítulo~\ref{cap:terceiroProjeto}. Nosso foco inicial será no Spring Boot, um \textit{framework} que nos ajudará a definir de forma transparente a configuração de diversos outros \textit{frameworks}, que usaremos com o objetivo de acelerar o desenvolvimento de aplicações em Java em geral que, no nosso caso, serão aplicações Web. O Spring Boot é uma baita mão na roda, pois ele cuidará de muitas coisas que teríamos que fazer manualmente, um processo tedioso e repleto de artimanhas para que as coisas funcionem da forma correta. Acredite, há alguns anos atrás, configurar diversos \textit{frameworks} e bibliotecas manualmente e, principalmente, fazer com que eles conversassem entre si era um verdadeiro pandemônio!

Agora que já sabemos mais ou menos com o que vamos lidar, nós podemos começar! Bora!!!
