\chapter*{Apresentação}
\epigraph{``\textit{Any fool can write code that a computer can understand. Good programmers write code that humans can understand}''.}{Martin Fowler}

\lettrine[lines=4, lhang=0.1, lraise=0, loversize=0.2, findent=0.1em]{\textcolor{corAzulTema}{P}}{rezado} aluno, seja bem-vindo! Este material contém diversos Capítulos, organizados de forma a guiá-lo no processo de fixação dos conteúdos aprendidos em aula, por meio de exercícios, desafios e de projetos práticos aplicados no contexto de desenvolvimento de aplicações Web em Java. A ordem dos Capítulos obedece a um caminho lógico que será empregado pelo professor no seu processo de aprendizagem, ou seja, a ordem dos Capítulos segue a ordem cronológica dos tópicos que serão apresentados, ensinados e treinados em laboratório.

\begin{wrapfigure}{R}{0.3\textwidth}
    \centering
    \includegraphics[width=0.25\textwidth]{imagens/david}
\end{wrapfigure}

Antes de começar, eu gostaria de me apresentar. Meu nome é David Buzatto e sou Bacharel em Sistemas de Informação pela Fundação de Ensino Octávio Bastos (2007), Mestre em Ciência da Computação pela Universidade Federal de São Carlos (2010) e Doutor em Biotecnologia pela Universidade de Ribeirão Preto (2017). Tenho interesse em algoritmos, estruturas de dados, compiladores, linguagens de programação, algoritmos em bioinformática e desenvolvimento de jogos eletrônicos. Atualmente sou professor efetivo do Instituto Federal de Educação, Ciência e Tecnologia de São Paulo (IFSP), câmpus São João da Boa Vista. A melhor forma de contatar é através do email \textcolor{blue}{\href{mailto:davidbuzatto.ifsp@gmail.com}{davidbuzatto.ifsp@gmail.com}}.

Para que você possa aproveitar a leitura deste documento de forma plena, vale a pena entender alguns padrões que foram utilizados neste texto. As três caixas apresentadas abaixo serão empregadas para mostrar, a você leitor, respectivamente, boas práticas de programação, conteúdos complementares para melhorar e aprofundar seu aprendizado e, por fim, itens que precisam ser tratados com cuidado ou que podem acarretar em erros comuns de programação.

\begin{boaPratica}
    Essa é uma caixa de ``Boa Prática''.
\end{boaPratica}

\begin{saibaMais}
    Essa é uma caixa de ``Saiba Mais''.
\end{saibaMais}

\begin{atencao}
    Essa é uma caixa de ``Atenção''.
\end{atencao}

Você notará que este documento foi escrito de forma quase coloquial, com o objetivo de conversar com você e não com o objetivo de ser um material de pesquisa ou acadêmico. É de suma importância que você resolva cada um dos exercícios básicos de cada Capítulo, visto que a utilização de uma linguagem de programação ou tecnologia, e mais importante ainda, a obtenção de maturidade no desenvolvimento de software, é ferramenta primordial para o seu sucesso profissional e intelectual na área da Computação.

Como última observação, vale ressaltar que este documento será constantemente atualizado, sendo assim, sempre obtenha a última versão no local indicado pelo professor. Espero que este material seja útil!

\vspace*{\fill}

\FloatBarrier
\begin{figure*}[!htbp]
    \centering
    \includegraphics[scale=0.1]{imagens/logoCC}
    \\
    Este trabalho está licenciado sob uma Licença Creative Commons Atribuição-NãoComercial-SemDerivações 4.0 Internacional. Para ver uma cópia desta licença, visite \textcolor{blue}{\url{http://creativecommons.org/licenses/by-nc-nd/4.0/}}.
\end{figure*}
\FloatBarrier
